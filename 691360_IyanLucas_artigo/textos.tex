%%%%%%%%%%%%%%%%%%%%%%%%%%%%%%%%%%%%%%%%%%%%%%%%%%%%%%%%%%%%%%%%%%%%%%%%%%%%%%%%%%%%%%%%%%%%%%%%%%%%%%%
%%%%%%%%%%%%%% Template de Artigo Adaptado para Trabalho de Diplomação do ICEI %%%%%%%%%%%%%%%%%%%%%%%%
%% codificação UTF-8 - Abntex - Latex -  							     %%
%% Autor:    Fábio Leandro Rodrigues Cordeiro  (fabioleandro@pucminas.br)                            %% 
%% Co-autores: Prof. João Paulo Domingos Silva, Harison da Silva e Anderson Carvalho		     %%
%% Revisores normas NBR (Padrão PUC Minas): Helenice Rego Cunha e Prof. Theldo Cruz                  %%
%% Versão: 1.1     18 de dezembro 2015                                                               %%
%%%%%%%%%%%%%%%%%%%%%%%%%%%%%%%%%%%%%%%%%%%%%%%%%%%%%%%%%%%%%%%%%%%%%%%%%%%%%%%%%%%%%%%%%%%%%%%%%%%%%%%
\section{\esp Definir e Categorizar}
\subsection{ASIC (Aplication Specific IC)}
% Os ASICs ou CIs customizados, se caracterizam principalmente pela
% necessidade de um processo de fabricação especial, que requer máscaras
% específicas para cada projeto. Essa característica acarreta em altos custos de projeto
% e um longo tempo de desenvolvimento para esse tipo de implementação, em caso de
% grandes implementações esse alto custo é amortizado. 
Os ASICs são circuitos integrados, analógico ou digital, que são feitos especificamente para algum propósito, desta forma eles são fabricados de forma específica, por isso tem um alto custo de projeto.
Por ser produzido de forma específica e ele também precisa de um tempo muito grande de produção e implementação.
Porém, por serem específicos, eles tem um desempenho muito alto e um baixo consumo de energia.



\subsection{ASSP (Aplication-Specific Standard Parts)}
São circuitos integrados similares aos ASIC, mas com uma função mais abrangente, visto que não são tão específicos por usarem partes padronizadas.
Desta forma é usado em diferentes projetos, então o custo é menor, mas o consumo e desempenho são inferiores aos ASIC.

% Circuitos integrados para uma aplicação específica padrão ASSPs (Application-specific standard parts) são projetados e implementados exatamente da mesma forma que ASICs.
% Isto não é surpreendente, pois são essencialmente a mesma coisa.
% A única diferença é que um ASSP é um dispositivo de propósito mais geral e é usado por diferentes sistemas em diferentes projetos.
% Por exemplo, um chip de interface USB pode ser classificado como um ASSP.
\subsection{SPLD (Simple PLD)}
São circuitos lógicos baseados em portas OR/AND com um baixo custo e alto desempenho, podendo ter flip-flops na saída.
As portas AND podem ser programadas, mas as OR são fixas.
Podem ser de dois tipos:
\begin{itemize}
	\item Programmable Array Logic (PAL): São SPLDs geralmente programáveis somente uma vez
	\item Generic Array Lociv (GAL): São SPLDs que podem ser reprogramáveis.
\end{itemize}


% baixo custo e alto desempenho
% Capacidade reduzida
\subsection{CPLD (Dispositivo de Lógica Programável Compléxa)}
Um CPLD é um dispositivo lógico programável introduzido pela Altera Corp. composto por vários SPLDs em um único chip com uma interconexão programável entre eles.
\par É um PLD que possui alta durabilidade, alta versatilidade com uma reconfiguração de até um milhão de vezes. 
Além disso, é um PLD com um baixo custo comparado aos outros PLD.

% Alta durabilidade mesmo em ambientes agressivos
% Alta versatilidade, pode ser reconfigurado cerca de 1M vezes
% Baixo custo
\subsection{SOC (System-on-Chips)}
Os SOCs são um sistema de chips de silício que contém um ou mais núcleos de microprocessadores, microcontroladores e processadores digitais de sinais.
Podem ser incluídos outros "módulos" como memória, aceleradores, periféricos, entre outros.
Se um ASIC ou um ASSP possui um ou mais processadores, então eles são um SOC.
Ele oferece alto desempenho e baixo custo assim como um ASIC e ASSP.

\subsection{FPGA (Field Programmable Gate Array)}
Os FPGAs são uma matriz de células configuráveis conectados entre si por uma conexão também programável, podendo prover uma capacidade lógica elevada.
Os primeiros foram feitos pela Xilinx Inc. em 1984, com duas linhas, a Spartan e Virtex, e possuia uma arquitetura simples, entretanto versões atuais já são mais complexas e possuem uma grande capacidade de configurações para o hardware desempenhar um grande leque de funções digitais.
Possui um tempo de projeto reduzido e um custo mais baixo.
Tem a possibilidade de implementar algorítmos paralelos, processando uma quantidade maior de dados de forma mais rápida e eficiente do que as SOC.


% é um grande arranjo de celulas configuraveis em um chip
% primeiro FPGA Xilinx inc. 1984
% reduzido tempo de projeto e relativo baixo custo
% blocos logicos , blocos I/O, chaves de interconexção
\section{\esp Diferenciar}
%  PROM & PLA & PAL \\
\subsection{PROM (Programmable Read Only Memory)}
É uma memória programável de só leitura, mas só pode ser programada uma vez, por causa do "rebentamento" dos fusíveis.
A PROM sai de fábrica com todos os fusíveis saindo 1, depois que ocorre a queima programada, o fusível passa 0.
Elas são utilizados para armazenamento permanente em programas, sendo largamente utilizadas em microcontroladores.

\subsection{PLA (Programmable Logic Arrays)}
Os Programmable Logic Arrays foi o primeiro dispositivo desenvolvido pra implementar funções logicas definidas.
São feitos por "arrays" de portas AND e OR sendo que as duas são programáveis, fazendo ser muito versátil.
Apesar disto o seu custo de projeto é elevado, usado em menor número.

% Introduzido no mercado pela Philips na década de 70, o PLA (Programmable
% Logic Array) foi o primeiro dispositivo desenvolvido para implementar funções lógicas
% definidas. Um PLA é constituído por arranjos AND e OR, onde ambos são
% programáveis. Por esse motivo o PLA é adequado para a implementação de funções
% na forma de soma de produtos, sendo muito versátil, pois tanto os arranjos AND
% como os OR podem ter muitas entradas. Porém, ela não foi bem aceita pelos
% projetistas pelo seu alto custo.
% A PLA possui ambos os planos programáveis.

\subsection{PAL (Programmable Array Logic)}
São SPLDs geralmente programáveis somente uma vez com um "array" de OR fixo.
São simples de fabricar, baratos e possuem alto desempenho.

\subsection{Comparativo}
\begin{center}
	\centering
	 \textbf{Quadro 1 - PAL x PLA}\\
% 	\vspace{-0.3cm} % espaço entre titulo e tabela
  \label{quadro1}
  \begin{tabular}{|c|c|c|} \hline
	\multicolumn{1}{|c|}{\textbf{}} &	\multicolumn{1}{|c|}{\textbf{PAL}} & \multicolumn{1}{|c|}{\textbf{PLA}} \\ 	  
  \hline \textbf{Porta AND programável} & NÃO & SIM\\ 
   \hline \textbf{Porta OR programável}  & SIM & SIM \\ 
  \hline \textbf{Custo} & BARATO & CARO \\ 
  \hline \textbf{Desempenho} & ALTO & ALTO \\ 
  \hline \textbf{Versátil} & NÃO & SIM \\ 
  \hline \textbf{Reprogramável} & SIM & NÃO \\ 
  \hline
\end{tabular}
\vspace{0.1cm} 
{\footnotesize\\ \textbf{Fonte: Dados da pesquisa}}
\end{center}


\section{\esp CLPD x FPGA}
Ambos os circuitos eletrônicos são usados amplamente na indústria, entretanto pequenos detalhes os diferenciam e tornam cada um melhor em certas situações.
Os CPLDs são excelentes escolhas quando o sistema é pequeno, ou precisa de respostas mais rápidas sem muita versatilidade.
Entretanto, quando o sistema toma proporções maiores, a escolha boa é o FPGA, visto que ele pode oferecer mais conexões, sendo um \textit{jack of all trades}. 
Porém, pelo seu custo elevado, dependendo do módulo, pode se utilizar ambos, em um sistema misto, o que seria ótimo, visto que o CLPD pode oferecer rapidez e custo mais baixo, enquanto o FPGA pode desenvolver múltiplas funções específicas, deixando os Trabalhos mais ordinários para os CLPDs.
Em suma, utilizar vários CLPDs e situacionais FPGAs traz um balanço de custo benefício e desempenho interessante para o sistema.\\
\begin{minipage}{\textwidth}
\begin{center}
\textbf{Quadro 2 - CLPD x FPGA}\\
% 	\vspace{-0.3cm} % espaço entre titulo e tabela
  \label{quadro2}
  \begin{tabular}{|c|c|c|} \hline
	\multicolumn{1}{|c|}{\textbf{}} &	\multicolumn{1}{|c|}{\textbf{CLPD}} & \multicolumn{1}{|c|}{\textbf{FPGA}} \\ 	  
  \hline \textbf{Módulos Lógicos} & PAL, GAL, PLA & CLB, IOB, \textit{Switch Matrix}\footnote{Muito menores que os módulos do CLPD} \\ 
   \hline \textbf{Memória}  & Não-volátil (Flash ou EEPROM) & Volátil (Baseado em RAM)\footnote{Modelos mais recentes já possuem memória não-volátil, apesar de serem bem específicos e caros utilizando EEPROM ou baseados em flash.} \\ 
   \hline \textbf{Blocos lógicos} & Até 10.000 & Até 100.000 \\ 
   \hline \textbf{Delay} &Determinístico e Menor\footnote{CLPDs são simples e possuem menos conectores, por isso, possui menor tempo de análise de delay }& Não determinístico\footnote{Pelo FPGA ter uma natureza complexa, os vários módulos e o tamanho podem tornar a predição do delay praticamente não determinística. Entretanto já possuem módulos caros fornecidos por fabricantes que auxiliam nessa tarefa. }\\ 
  \hline \textbf{\textit{Idle Power}} & Baixo & Alto \\ 
  \hline \textbf{Reprogramável} & Sim\footnote{ Porém tem que rebootar o circuito e reconfigurá-lo }& Sim\footnote{ Pode reconfigurar enquanto o processo está rodando, não necessita de rebootar }\\ 
  \hline
\end{tabular}
\vspace{0.1cm} 
{\footnotesize\\ \textbf{Fonte: Dados da pesquisa}}
\end{center}
\end{minipage}

% pra mudar o cirquito ou fazer alteração o FPGA pode fazer isso até rodando, o CLPD precisa desligar, etc.
% FPGA caro para projetos simples CLPD barato
% FPGA não dá pra predizer o delay mas o CLPD tem um delay muito baixo por isso é melhor pra sincronização de data paralela 
% idle power consumption menor no cpdl maior no FPGA
% CPLD tem memória não volatil / FPGA tem memória voláril (novos modelos tem memória não volátil)
% CPLD é melhor para cirquitos pequenos
% FPGA pode ter até 100.000 blocos lógicos, enquanto CLPD pode ter até 10.000
% FPGA não usa PAL, GAL, PLA. no lugar usa modulos logicos muito menores com uma interconexção programável pra conectar dentro do bloco logico configurável.



